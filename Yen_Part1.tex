% Options for packages loaded elsewhere
\PassOptionsToPackage{unicode}{hyperref}
\PassOptionsToPackage{hyphens}{url}
%
\documentclass[
]{article}
\usepackage{amsmath,amssymb}
\usepackage{lmodern}
\usepackage{iftex}
\ifPDFTeX
  \usepackage[T1]{fontenc}
  \usepackage[utf8]{inputenc}
  \usepackage{textcomp} % provide euro and other symbols
\else % if luatex or xetex
  \usepackage{unicode-math}
  \defaultfontfeatures{Scale=MatchLowercase}
  \defaultfontfeatures[\rmfamily]{Ligatures=TeX,Scale=1}
\fi
% Use upquote if available, for straight quotes in verbatim environments
\IfFileExists{upquote.sty}{\usepackage{upquote}}{}
\IfFileExists{microtype.sty}{% use microtype if available
  \usepackage[]{microtype}
  \UseMicrotypeSet[protrusion]{basicmath} % disable protrusion for tt fonts
}{}
\makeatletter
\@ifundefined{KOMAClassName}{% if non-KOMA class
  \IfFileExists{parskip.sty}{%
    \usepackage{parskip}
  }{% else
    \setlength{\parindent}{0pt}
    \setlength{\parskip}{6pt plus 2pt minus 1pt}}
}{% if KOMA class
  \KOMAoptions{parskip=half}}
\makeatother
\usepackage{xcolor}
\IfFileExists{xurl.sty}{\usepackage{xurl}}{} % add URL line breaks if available
\IfFileExists{bookmark.sty}{\usepackage{bookmark}}{\usepackage{hyperref}}
\hypersetup{
  pdftitle={Part1 - sicss-howard-mathematica-R-2022},
  pdfauthor={Evan Muzzall, SSDS, Stanford University, https://ssds.stanford.edu/},
  hidelinks,
  pdfcreator={LaTeX via pandoc}}
\urlstyle{same} % disable monospaced font for URLs
\usepackage[margin=1in]{geometry}
\usepackage{color}
\usepackage{fancyvrb}
\newcommand{\VerbBar}{|}
\newcommand{\VERB}{\Verb[commandchars=\\\{\}]}
\DefineVerbatimEnvironment{Highlighting}{Verbatim}{commandchars=\\\{\}}
% Add ',fontsize=\small' for more characters per line
\usepackage{framed}
\definecolor{shadecolor}{RGB}{248,248,248}
\newenvironment{Shaded}{\begin{snugshade}}{\end{snugshade}}
\newcommand{\AlertTok}[1]{\textcolor[rgb]{0.94,0.16,0.16}{#1}}
\newcommand{\AnnotationTok}[1]{\textcolor[rgb]{0.56,0.35,0.01}{\textbf{\textit{#1}}}}
\newcommand{\AttributeTok}[1]{\textcolor[rgb]{0.77,0.63,0.00}{#1}}
\newcommand{\BaseNTok}[1]{\textcolor[rgb]{0.00,0.00,0.81}{#1}}
\newcommand{\BuiltInTok}[1]{#1}
\newcommand{\CharTok}[1]{\textcolor[rgb]{0.31,0.60,0.02}{#1}}
\newcommand{\CommentTok}[1]{\textcolor[rgb]{0.56,0.35,0.01}{\textit{#1}}}
\newcommand{\CommentVarTok}[1]{\textcolor[rgb]{0.56,0.35,0.01}{\textbf{\textit{#1}}}}
\newcommand{\ConstantTok}[1]{\textcolor[rgb]{0.00,0.00,0.00}{#1}}
\newcommand{\ControlFlowTok}[1]{\textcolor[rgb]{0.13,0.29,0.53}{\textbf{#1}}}
\newcommand{\DataTypeTok}[1]{\textcolor[rgb]{0.13,0.29,0.53}{#1}}
\newcommand{\DecValTok}[1]{\textcolor[rgb]{0.00,0.00,0.81}{#1}}
\newcommand{\DocumentationTok}[1]{\textcolor[rgb]{0.56,0.35,0.01}{\textbf{\textit{#1}}}}
\newcommand{\ErrorTok}[1]{\textcolor[rgb]{0.64,0.00,0.00}{\textbf{#1}}}
\newcommand{\ExtensionTok}[1]{#1}
\newcommand{\FloatTok}[1]{\textcolor[rgb]{0.00,0.00,0.81}{#1}}
\newcommand{\FunctionTok}[1]{\textcolor[rgb]{0.00,0.00,0.00}{#1}}
\newcommand{\ImportTok}[1]{#1}
\newcommand{\InformationTok}[1]{\textcolor[rgb]{0.56,0.35,0.01}{\textbf{\textit{#1}}}}
\newcommand{\KeywordTok}[1]{\textcolor[rgb]{0.13,0.29,0.53}{\textbf{#1}}}
\newcommand{\NormalTok}[1]{#1}
\newcommand{\OperatorTok}[1]{\textcolor[rgb]{0.81,0.36,0.00}{\textbf{#1}}}
\newcommand{\OtherTok}[1]{\textcolor[rgb]{0.56,0.35,0.01}{#1}}
\newcommand{\PreprocessorTok}[1]{\textcolor[rgb]{0.56,0.35,0.01}{\textit{#1}}}
\newcommand{\RegionMarkerTok}[1]{#1}
\newcommand{\SpecialCharTok}[1]{\textcolor[rgb]{0.00,0.00,0.00}{#1}}
\newcommand{\SpecialStringTok}[1]{\textcolor[rgb]{0.31,0.60,0.02}{#1}}
\newcommand{\StringTok}[1]{\textcolor[rgb]{0.31,0.60,0.02}{#1}}
\newcommand{\VariableTok}[1]{\textcolor[rgb]{0.00,0.00,0.00}{#1}}
\newcommand{\VerbatimStringTok}[1]{\textcolor[rgb]{0.31,0.60,0.02}{#1}}
\newcommand{\WarningTok}[1]{\textcolor[rgb]{0.56,0.35,0.01}{\textbf{\textit{#1}}}}
\usepackage{graphicx}
\makeatletter
\def\maxwidth{\ifdim\Gin@nat@width>\linewidth\linewidth\else\Gin@nat@width\fi}
\def\maxheight{\ifdim\Gin@nat@height>\textheight\textheight\else\Gin@nat@height\fi}
\makeatother
% Scale images if necessary, so that they will not overflow the page
% margins by default, and it is still possible to overwrite the defaults
% using explicit options in \includegraphics[width, height, ...]{}
\setkeys{Gin}{width=\maxwidth,height=\maxheight,keepaspectratio}
% Set default figure placement to htbp
\makeatletter
\def\fps@figure{htbp}
\makeatother
\setlength{\emergencystretch}{3em} % prevent overfull lines
\providecommand{\tightlist}{%
  \setlength{\itemsep}{0pt}\setlength{\parskip}{0pt}}
\setcounter{secnumdepth}{-\maxdimen} % remove section numbering
\ifLuaTeX
  \usepackage{selnolig}  % disable illegal ligatures
\fi

\title{Part1 - sicss-howard-mathematica-R-2022}
\author{Evan Muzzall, SSDS, Stanford University,
\url{https://ssds.stanford.edu/}}
\date{18 June 2022}

\begin{document}
\maketitle

{
\setcounter{tocdepth}{2}
\tableofcontents
}
\hypertarget{computational-social-science-css-example-research-workflow}{%
\section{Computational Social Science (CSS) example research
workflow}\label{computational-social-science-css-example-research-workflow}}

Scientific research is necessary because it provides solutions to many
problems, from those found in everyday life to the mysteries of
existence. It is used to build knowledge, teach, learn, and increase
awareness through cutting edge technologies and techniques that create
new ideas and that are born out of creativity and curiosity. ``Social''
science research focuses on humans, their interactions, and the
societies they exist within and relies on data - forms of information -
for it to work.

While it is fun to discuss the postmodern constructs for the definition
of data, in the context of this workshop it is defined as quantitative
and qualitative representations of information generally found in the
form of numbers, text, images, video, and audio.

How then, might we approach a data-intensive computational social
science research project? The example below is a simplified
representation.

\begin{figure}
\centering
\includegraphics{img/workflow.png}
\caption{workflow}
\end{figure}

\hypertarget{preparation}{%
\subsection{Preparation}\label{preparation}}

\begin{itemize}
\tightlist
\item
  \textbf{Read} the relevant literature in your field. What questions
  are those authors asking, and in what directions do they suggest
  future research should go? What are the limitations and other
  problems?
\item
  \textbf{Ask a question} forged from your knowledge of the literature.
  You can cobble together a virtually endless supply of research
  questions with some creative thinking and a critical perspective.
\item
  \textbf{Design} your research framework. What type of question(s) are
  you asking, what are the assumptions of the data you seek and what are
  your expected outcomes?
\item
  Define \textbf{milestones.} When is your final deadline? Divide your
  project into subparts. Completing smaller goals will not only give you
  a sense of progress, but will also allow you to see how the parts
  connect together and will improve the quality of yoru work.
\end{itemize}

\hypertarget{data}{%
\subsection{Data}\label{data}}

\begin{itemize}
\tightlist
\item
  \textbf{Acquire} data. All projects require some form of data,
  theoretical musings aside, whether you obtain it from a government or
  other organization, scrape it from the web, or collect it yourself.
\item
  \textbf{Import} the data into R. As you can see, there is a lot of
  work to be done even before the computational part! However, do not
  let this discourage you from experimenting with R.
\item
  \textbf{Wrangle} your data when necessary. Ensuring its quality and
  validation is essential because the data will determine your results
  along with their interpretation. Inconsistencies and other errors are
  inevitable, especially when compiling data from multiple sources.
  Close visual inspection for errors is always a good first step, but
  you will likely spend much of your time fixing structural errors,
  standardizing names, handling missing values, dealing with outliers,
  and removing duplicate observations. You should always document your
  data wrangling process not only to inform your audience, but also in
  case you need to refer to back to it in the future.
\end{itemize}

\hypertarget{exploration}{%
\subsection{Exploration}\label{exploration}}

\begin{itemize}
\tightlist
\item
  \textbf{Summarize} your data. A common misconception is that a
  researcher must incorporate complicated methods and artificial
  intelligence to produce meaningful results. However, often-overlooked
  summary statistics can not only help cross-check your data wrangling,
  but can also elucidate patterns that can influence your assumptions
  and expectations.
\item
  \textbf{Visualize} aspects of your data. The goal of data
  visualization is to communicate some aspect of your data to an
  audience in a clear and concise way.
\end{itemize}

\hypertarget{analysis}{%
\subsection{Analysis}\label{analysis}}

\begin{itemize}
\tightlist
\item
  \textbf{Evaluate} specific facets of the data now that you have built
  a foundation to do so.
\end{itemize}

\hypertarget{presentation}{%
\subsection{Presentation}\label{presentation}}

\begin{itemize}
\tightlist
\item
  Effective communication of your subject matter is just as important as
  knowing your stuff. Whether you are presenting a slideshow to
  colleagues, poster, conference talk, webinar, or submitting a
  manuscript for peer-review, effective communication of your subject
  matter is just as important as knowing your stuff and takes a lot of
  practice.
\end{itemize}

\hypertarget{why-r}{%
\section{Why R?}\label{why-r}}

Data need to be imported, wrangled, explored, and analyzed in order to
make contributions to legitimate bodies of knowledge.

R is expert at these computational parts of the research process, It is
a free (to use, not necessarily to develop!) and open programming
language that helps you teach a computer to perform a ``data science''
task. Note that R was designed to do statistics/data science from its
conception and has evolved that way.

\hypertarget{what-is-rstudio}{%
\subsection{What is RStudio?}\label{what-is-rstudio}}

RStudio is the graphical user interface/environment that we will program
R inside of. It is really helpful for staying organized and offers much
more functionality than the R language provides by itself. RStudio is
also highly customizable.
\href{https://support.rstudio.com/hc/en-us/articles/200549016-Customizing-the-RStudio-IDE}{Read
the customization guide by clicking here}.

\hypertarget{source-files}{%
\subsubsection{Source files}\label{source-files}}

So, where do we type instructions for these data related tasks

\hypertarget{script}{%
\paragraph{Script}\label{script}}

Scripts are simply plain text files with a .R file extension. Click
\texttt{File\ -\/-\textgreater{}\ New\ File\ -\/-\textgreater{}\ R\ Script}
to open one. In R scripts: - Code is entered normally - Hashtags
\texttt{\#} are used to comment your code, or make notes to yourself and
others

\hypertarget{r-markdown-file}{%
\paragraph{R Markdown file}\label{r-markdown-file}}

R Markdown files utilize markdown, a lightweight markup language, to
allow you to intersperse code and formattable text.
\href{https://rmarkdown.rstudio.com/lesson-1.html}{Read the guide} to
learn how to format your R Markdown files.

Enter code in \href{https://rmarkdown.rstudio.com/lesson-3.html}{chunks}
and press \texttt{Control\ +\ Enter} to run a line of code, run multiple
highlighted lines of code, or press the green ``Play'' button on the
right side to run the entire chunk.

\begin{Shaded}
\begin{Highlighting}[]
\DecValTok{2} \SpecialCharTok{+} \DecValTok{2}
\end{Highlighting}
\end{Shaded}

\begin{verbatim}
## [1] 4
\end{verbatim}

\begin{Shaded}
\begin{Highlighting}[]
\FunctionTok{cos}\NormalTok{(pi)}
\end{Highlighting}
\end{Shaded}

\begin{verbatim}
## [1] -1
\end{verbatim}

\hypertarget{navigating-rstudio}{%
\subsubsection{Navigating RStudio}\label{navigating-rstudio}}

\begin{figure}
\centering
\includegraphics{img/navigate_rstudio.png}
\caption{navigate\_rstudio}
\end{figure}

\hypertarget{r-projects}{%
\subsection{R projects}\label{r-projects}}

R Projects files (.Rproj) are associated with your
\href{http://www.sthda.com/english/wiki/running-rstudio-and-setting-up-your-working-directory-easy-r-programming}{\textbf{working
directory}} (the location on your computer that your actions are
relevant to) and are great for staying organized by organizing your
project and its context into a single location. The working directory is
set and retrieved via a \textbf{file path}.

\begin{quote}
You can change the working directory location with the \texttt{setwd()}
function, or by clicking ``Session'' --\textgreater{} ``Set Working
Directory'' --\textgreater{} ``Choose Directory'' from the top toolbar
menu. However,
\end{quote}

\href{https://support.rstudio.com/hc/en-us/articles/200526207-Using-RStudio-Projects}{Learn
more about setting up an R Project by clicking here.}

\hypertarget{basic-building-blocks-of-r-programming}{%
\section{Basic building blocks of R
programming}\label{basic-building-blocks-of-r-programming}}

You will find that no matter how complex of a task you are faced with,
you will pretty much always rely on the same basic building blocks to
accomplish it. However, it is up to you to arrange them in specific ways
to accomplish something creative and meaningful.

\hypertarget{objects-functions-and-arguments}{%
\subsection{Objects, functions, and
arguments}\label{objects-functions-and-arguments}}

Everything that exists in R is an \textbf{object} or some piece of
information/data that can be manipulated.

Almost everything that happens is a \textbf{function call}, or a command
that performs an action on a ``thing''. Functions always contain
trailing round parentheses \texttt{()}.

These ``things'' are called \textbf{arguments} and are defined in
function creation as \textbf{parameters.}

\hypertarget{variable-assignment}{%
\subsection{Variable assignment}\label{variable-assignment}}

Variables (which are also objects) are how data are saved in R's memory,
which lives in a physical location on your computer. These are just
placeholders and can contain virtually anything: values, mathematical
expressions, text, functions, or entire datasets.

\texttt{\textless{}-} is the assignment operator. It assigns values on
the right to objects on the left. So, after executing
\texttt{x\ \textless{}-\ 5}, the value of \texttt{x} is \texttt{5}.

You can read this in plain language as ``x is defined as 5'', ``5 is
assigned to x'', or most simply as ``x is equal to 5''.

This performs the assignment step. Note that the variable now appears on
your ``Environment'' tab.

Type the name of the variable and run it (``call'' the variable) to show
it on the screen.

\begin{Shaded}
\begin{Highlighting}[]
\NormalTok{x }\OtherTok{\textless{}{-}} \DecValTok{5}
\NormalTok{x}
\end{Highlighting}
\end{Shaded}

\begin{verbatim}
## [1] 5
\end{verbatim}

\hypertarget{data-types-numeric-character-logical-integer-factor}{%
\subsection{Data types: numeric, character, logical, integer,
factor}\label{data-types-numeric-character-logical-integer-factor}}

Like everything else in R, data have a class (type) associated with it
which determines how we can manipulate it. Use the \texttt{class()}
function to find out. In this case, the variable \texttt{x} is the
argument.

We will talk about five basic types:

\begin{enumerate}
\def\labelenumi{\arabic{enumi}.}
\item
  Numeric: decimals; the default class for all numbers in R
\item
  Character: text, always wrapped in quotations \texttt{"\ "} (single or
  double are fine, be consistent)
\item
  Logical: TRUE and FALSE; stored internally as 1 and 0 and as such take
  on mathematical properties; useful for turning function parameters
  ``on'' or ``off'', as well as for data subsetting (see below).
\item
  Integer: positive and negative whole numbers, including zero
\item
  Factor: categorical data
\item
  Numeric class
\end{enumerate}

\begin{Shaded}
\begin{Highlighting}[]
\FunctionTok{class}\NormalTok{(x)}
\end{Highlighting}
\end{Shaded}

\begin{verbatim}
## [1] "numeric"
\end{verbatim}

\begin{enumerate}
\def\labelenumi{\arabic{enumi}.}
\setcounter{enumi}{1}
\tightlist
\item
  Character
\end{enumerate}

\begin{Shaded}
\begin{Highlighting}[]
\CommentTok{\# use a hashtag within a code chunk to comment your code}
\CommentTok{\# note the use of the underscore to represent a space in the variable name}
\NormalTok{my\_name }\OtherTok{\textless{}{-}} \StringTok{"Evan"}
\NormalTok{my\_name}
\end{Highlighting}
\end{Shaded}

\begin{verbatim}
## [1] "Evan"
\end{verbatim}

\begin{Shaded}
\begin{Highlighting}[]
\FunctionTok{class}\NormalTok{(my\_name)}
\end{Highlighting}
\end{Shaded}

\begin{verbatim}
## [1] "character"
\end{verbatim}

\begin{enumerate}
\def\labelenumi{\arabic{enumi}.}
\setcounter{enumi}{2}
\tightlist
\item
  Logical
\end{enumerate}

In R, TRUE is 1 FALSE is 0

\begin{Shaded}
\begin{Highlighting}[]
\ConstantTok{TRUE} \SpecialCharTok{+} \DecValTok{1}
\end{Highlighting}
\end{Shaded}

\begin{verbatim}
## [1] 2
\end{verbatim}

\begin{Shaded}
\begin{Highlighting}[]
\ConstantTok{FALSE} \SpecialCharTok{{-}} \DecValTok{1}
\end{Highlighting}
\end{Shaded}

\begin{verbatim}
## [1] -1
\end{verbatim}

\begin{Shaded}
\begin{Highlighting}[]
\CommentTok{\# is 3 greater than 4?}
\DecValTok{3} \SpecialCharTok{\textgreater{}} \DecValTok{4}
\end{Highlighting}
\end{Shaded}

\begin{verbatim}
## [1] FALSE
\end{verbatim}

\begin{Shaded}
\begin{Highlighting}[]
\CommentTok{\# is 4 less than or equal to 4?}
\DecValTok{4} \SpecialCharTok{\textless{}=} \DecValTok{4}
\end{Highlighting}
\end{Shaded}

\begin{verbatim}
## [1] TRUE
\end{verbatim}

\begin{Shaded}
\begin{Highlighting}[]
\CommentTok{\# is "Apple" equal to "apple" (hint: R is case sensitive!)}
\StringTok{"Apple"} \SpecialCharTok{==} \StringTok{"apple"}
\end{Highlighting}
\end{Shaded}

\begin{verbatim}
## [1] FALSE
\end{verbatim}

\begin{Shaded}
\begin{Highlighting}[]
\CommentTok{\# is "Apple" not equal to "Apple"}
\StringTok{"Apple"} \SpecialCharTok{!=} \StringTok{"Apple"}
\end{Highlighting}
\end{Shaded}

\begin{verbatim}
## [1] FALSE
\end{verbatim}

\begin{enumerate}
\def\labelenumi{\arabic{enumi}.}
\setcounter{enumi}{3}
\tightlist
\item
  Integer
\end{enumerate}

We can use various ``as dot'' functions to convert data types. To
convert numeric to integer class for example, we could type:

\begin{Shaded}
\begin{Highlighting}[]
\NormalTok{y }\OtherTok{\textless{}{-}} \FunctionTok{as.integer}\NormalTok{(x)}
\NormalTok{y}
\end{Highlighting}
\end{Shaded}

\begin{verbatim}
## [1] 5
\end{verbatim}

\begin{Shaded}
\begin{Highlighting}[]
\FunctionTok{class}\NormalTok{(y)}
\end{Highlighting}
\end{Shaded}

\begin{verbatim}
## [1] "integer"
\end{verbatim}

\begin{enumerate}
\def\labelenumi{\arabic{enumi}.}
\setcounter{enumi}{4}
\tightlist
\item
  Factor
\end{enumerate}

Other ``as dot'' functions exist as well: \texttt{as.character()},
\texttt{as.numeric()}, and \texttt{as.factor()} to name a few:

\begin{Shaded}
\begin{Highlighting}[]
\NormalTok{school }\OtherTok{\textless{}{-}} \StringTok{"Stanford University"}
\NormalTok{school}
\end{Highlighting}
\end{Shaded}

\begin{verbatim}
## [1] "Stanford University"
\end{verbatim}

\begin{Shaded}
\begin{Highlighting}[]
\FunctionTok{class}\NormalTok{(school)}
\end{Highlighting}
\end{Shaded}

\begin{verbatim}
## [1] "character"
\end{verbatim}

\begin{Shaded}
\begin{Highlighting}[]
\CommentTok{\# convert to factor}
\NormalTok{school\_fac }\OtherTok{\textless{}{-}} \FunctionTok{as.factor}\NormalTok{(school)}
\NormalTok{school\_fac}
\end{Highlighting}
\end{Shaded}

\begin{verbatim}
## [1] Stanford University
## Levels: Stanford University
\end{verbatim}

\begin{Shaded}
\begin{Highlighting}[]
\FunctionTok{class}\NormalTok{(school\_fac)}
\end{Highlighting}
\end{Shaded}

\begin{verbatim}
## [1] "factor"
\end{verbatim}

It might seem difficult keeping track of variables you define, but
remember they are listed in RStudio's ``Environment'' tab. You can also
type \texttt{ls()} to print them to the console.

\begin{Shaded}
\begin{Highlighting}[]
\FunctionTok{ls}\NormalTok{()}
\end{Highlighting}
\end{Shaded}

\begin{verbatim}
## [1] "my_name"    "school"     "school_fac" "x"          "y"
\end{verbatim}

Remove a single variable with \texttt{rm()}

\begin{Shaded}
\begin{Highlighting}[]
\FunctionTok{rm}\NormalTok{(my\_name)}
\NormalTok{my\_name }\CommentTok{\# Error}
\end{Highlighting}
\end{Shaded}

\begin{quote}
Remember to use autocomplete when typing a function or variable name,
since there is great potential for humans to make syntactical errors
\end{quote}

Alternatively, you can wipe your Environment clean by clicking the
yellow broom icon on the Environment tab or by typing

\begin{Shaded}
\begin{Highlighting}[]
\FunctionTok{rm}\NormalTok{(}\AttributeTok{list =} \FunctionTok{ls}\NormalTok{())}
\end{Highlighting}
\end{Shaded}

If your environment gets too messy, pressy \texttt{ctrl\ +\ l} to return
the prompt to the top and make it more readalble. This also makes
scrolling through your output much easier!

\begin{quote}
To completely restart your R session, click ``Session'' --\textgreater{}
``Restart R'' from the top toolbar menu.
\end{quote}

\hypertarget{data-structures-vector-data-frame}{%
\subsection{Data structures: vector, data
frame}\label{data-structures-vector-data-frame}}

If saving one piece of data in a variable is good, saving many is
better. Use the \texttt{c()} function to combine multiple pieces of data
into a \textbf{vector}, which is an ordered group of the same type of
data.

We can nest the \texttt{c()} function inside of ``as dot'' functions to
create vectors of different types.

\begin{Shaded}
\begin{Highlighting}[]
\CommentTok{\# example numeric (default) vector}
\NormalTok{traffic\_stops }\OtherTok{\textless{}{-}} \FunctionTok{c}\NormalTok{(}\DecValTok{8814}\NormalTok{, }\DecValTok{9915}\NormalTok{, }\DecValTok{9829}\NormalTok{, }\DecValTok{10161}\NormalTok{, }\DecValTok{6810}\NormalTok{, }\DecValTok{8991}\NormalTok{)}

\CommentTok{\# Integer, logical, and factor vectors}
\NormalTok{city }\OtherTok{\textless{}{-}} \FunctionTok{as.factor}\NormalTok{(}\FunctionTok{c}\NormalTok{(}\StringTok{"SF"}\NormalTok{, }\StringTok{"DC"}\NormalTok{, }\StringTok{"DC"}\NormalTok{, }\StringTok{"DC"}\NormalTok{, }\StringTok{"SF"}\NormalTok{, }\StringTok{"SF"}\NormalTok{))}
\NormalTok{year }\OtherTok{\textless{}{-}} \FunctionTok{as.integer}\NormalTok{(}\FunctionTok{c}\NormalTok{(}\DecValTok{2000}\NormalTok{, }\DecValTok{2000}\NormalTok{, }\DecValTok{2001}\NormalTok{, }\DecValTok{2002}\NormalTok{, }\DecValTok{2001}\NormalTok{, }\DecValTok{2002}\NormalTok{))}

\CommentTok{\# Call these variables to print them to the screen and check their class}
\NormalTok{traffic\_stops}
\end{Highlighting}
\end{Shaded}

\begin{verbatim}
## [1]  8814  9915  9829 10161  6810  8991
\end{verbatim}

\begin{Shaded}
\begin{Highlighting}[]
\NormalTok{city}
\end{Highlighting}
\end{Shaded}

\begin{verbatim}
## [1] SF DC DC DC SF SF
## Levels: DC SF
\end{verbatim}

\begin{Shaded}
\begin{Highlighting}[]
\NormalTok{year}
\end{Highlighting}
\end{Shaded}

\begin{verbatim}
## [1] 2000 2000 2001 2002 2001 2002
\end{verbatim}

\begin{Shaded}
\begin{Highlighting}[]
\FunctionTok{class}\NormalTok{(traffic\_stops)}
\end{Highlighting}
\end{Shaded}

\begin{verbatim}
## [1] "numeric"
\end{verbatim}

\begin{Shaded}
\begin{Highlighting}[]
\FunctionTok{class}\NormalTok{(city)}
\end{Highlighting}
\end{Shaded}

\begin{verbatim}
## [1] "factor"
\end{verbatim}

\begin{Shaded}
\begin{Highlighting}[]
\FunctionTok{class}\NormalTok{(year)}
\end{Highlighting}
\end{Shaded}

\begin{verbatim}
## [1] "integer"
\end{verbatim}

A \textbf{data frame} is an ordered group of \textbf{equal-length}
vectors.

\begin{quote}
More simply put, a data frame is a tabular data structure organized into
horizontal rows and vertical columns, i.e.~a spreadsheet! These are
often stored as comma separated values (.csv) files, or plain text where
commas are used to delineate column breaks and that look good in
spreadsheet programs like Microsoft Excel.
\end{quote}

We can assemble our three vectors from above into a data frame with the
\texttt{data.frame()} function.

\begin{Shaded}
\begin{Highlighting}[]
\NormalTok{police }\OtherTok{\textless{}{-}} \FunctionTok{data.frame}\NormalTok{(city, traffic\_stops, year)}
\NormalTok{police}
\end{Highlighting}
\end{Shaded}

\begin{verbatim}
##   city traffic_stops year
## 1   SF          8814 2000
## 2   DC          9915 2000
## 3   DC          9829 2001
## 4   DC         10161 2002
## 5   SF          6810 2001
## 6   SF          8991 2002
\end{verbatim}

\begin{Shaded}
\begin{Highlighting}[]
\FunctionTok{class}\NormalTok{(police)}
\end{Highlighting}
\end{Shaded}

\begin{verbatim}
## [1] "data.frame"
\end{verbatim}

\begin{Shaded}
\begin{Highlighting}[]
\CommentTok{\# display the compact structure of a data frame}
\FunctionTok{str}\NormalTok{(police)}
\end{Highlighting}
\end{Shaded}

\begin{verbatim}
## 'data.frame':    6 obs. of  3 variables:
##  $ city         : Factor w/ 2 levels "DC","SF": 2 1 1 1 2 2
##  $ traffic_stops: num  8814 9915 9829 10161 6810 ...
##  $ year         : int  2000 2000 2001 2002 2001 2002
\end{verbatim}

\begin{Shaded}
\begin{Highlighting}[]
\CommentTok{\# class = data.frame}
\CommentTok{\# 6 observations (rows)}
\CommentTok{\# 3 variables (columns, or vectors)}
\CommentTok{\# column names are preceded by the dollar sign }
\end{Highlighting}
\end{Shaded}

\hypertarget{challenge---dataframes}{%
\section{Challenge - dataframes}\label{challenge---dataframes}}

\begin{enumerate}
\def\labelenumi{\arabic{enumi}.}
\item
  Create a dataframe that contains 6 rows and 3 columns by following the
  instructions above.
\item
  Advanced: what is the difference between a data frame and a tidyverse
  tibble?
\end{enumerate}

\begin{Shaded}
\begin{Highlighting}[]
\DocumentationTok{\#\# YOUR CODE HERE}
\end{Highlighting}
\end{Shaded}

\hypertarget{indexing}{%
\subsection{Indexing}\label{indexing}}

A vector can be indexed (positionally referenced) by typing its name
followed by its index within square brackets. For example, if we want to
index just the first thing in the city vector, we could type

Note that R is a ``one-indexed'' programming language. This means that
counting anything starts at 1.

\begin{Shaded}
\begin{Highlighting}[]
\NormalTok{city[}\DecValTok{1}\NormalTok{]}
\end{Highlighting}
\end{Shaded}

\begin{verbatim}
## [1] SF
## Levels: DC SF
\end{verbatim}

\begin{Shaded}
\begin{Highlighting}[]
\CommentTok{\# or}
\NormalTok{police}\SpecialCharTok{$}\NormalTok{city[}\DecValTok{1}\NormalTok{]}
\end{Highlighting}
\end{Shaded}

\begin{verbatim}
## [1] SF
## Levels: DC SF
\end{verbatim}

If we want to return just the third thing in traffic\_stops, we would
type

\begin{Shaded}
\begin{Highlighting}[]
\NormalTok{traffic\_stops[}\DecValTok{3}\NormalTok{]}
\end{Highlighting}
\end{Shaded}

\begin{verbatim}
## [1] 9829
\end{verbatim}

\begin{Shaded}
\begin{Highlighting}[]
\CommentTok{\# or}
\NormalTok{police}\SpecialCharTok{$}\NormalTok{traffic\_stops[}\DecValTok{3}\NormalTok{]}
\end{Highlighting}
\end{Shaded}

\begin{verbatim}
## [1] 9829
\end{verbatim}

\hypertarget{subsetting-single-columns-with}{%
\subsection{\texorpdfstring{Subsetting single columns with
\texttt{\$}}{Subsetting single columns with \$}}\label{subsetting-single-columns-with}}

Note that columns are preceded by the dollar sign \texttt{\$}. You can
access a single column by typing the name of your data frame, the
\texttt{\$}, and then the column name. Note that autocomplete works for
much more than just function and variable names!

\begin{Shaded}
\begin{Highlighting}[]
\CommentTok{\# show just the column containing the number of traffic stops}
\NormalTok{police}\SpecialCharTok{$}\NormalTok{traffic\_stops}
\end{Highlighting}
\end{Shaded}

\begin{verbatim}
## [1]  8814  9915  9829 10161  6810  8991
\end{verbatim}

\begin{Shaded}
\begin{Highlighting}[]
\CommentTok{\# ... which can then easily be plugged into another function}
\FunctionTok{hist}\NormalTok{(police}\SpecialCharTok{$}\NormalTok{traffic\_stops)}
\end{Highlighting}
\end{Shaded}

\includegraphics{Yen_Part1_files/figure-latex/unnamed-chunk-16-1.pdf}

\hypertarget{subsetting-rows-and-columns-with-bracket-notation}{%
\subsection{\texorpdfstring{Subsetting rows and columns with bracket
notation
\texttt{{[},{]}}}{Subsetting rows and columns with bracket notation {[},{]}}}\label{subsetting-rows-and-columns-with-bracket-notation}}

This can also be extended to rows and columns using bracket notation
\texttt{{[},{]}}

Type the name of your data frame followed by square brackets with a
comma inbetween them.

Here, we can enter two indices: one for the rows (before the comma) and
one for the columns (after the comma) like this:
\texttt{{[}rows,\ cols{]}}

For example, if we want two columns, we cannot use the dollar sign
operator (since it only works for single columns), but we could type
either the indices or the columns names as a vector!

If either the row or column position is left blank, all rows/columns
will be retured becuase no subset was specified.

To subset the \texttt{police} with just the city name and number of
stops columns, type

\begin{Shaded}
\begin{Highlighting}[]
\NormalTok{city\_and\_stops }\OtherTok{\textless{}{-}}\NormalTok{ police[,}\FunctionTok{c}\NormalTok{(}\DecValTok{1}\NormalTok{,}\DecValTok{2}\NormalTok{)]}
\NormalTok{city\_and\_stops}
\end{Highlighting}
\end{Shaded}

\begin{verbatim}
##   city traffic_stops
## 1   SF          8814
## 2   DC          9915
## 3   DC          9829
## 4   DC         10161
## 5   SF          6810
## 6   SF          8991
\end{verbatim}

\begin{Shaded}
\begin{Highlighting}[]
\CommentTok{\# or, for consecutive sequences}
\NormalTok{city\_and\_stops }\OtherTok{\textless{}{-}}\NormalTok{ police[,}\DecValTok{1}\SpecialCharTok{:}\DecValTok{2}\NormalTok{]}
\NormalTok{city\_and\_stops}
\end{Highlighting}
\end{Shaded}

\begin{verbatim}
##   city traffic_stops
## 1   SF          8814
## 2   DC          9915
## 3   DC          9829
## 4   DC         10161
## 5   SF          6810
## 6   SF          8991
\end{verbatim}

\begin{Shaded}
\begin{Highlighting}[]
\CommentTok{\# or using variable names}
\NormalTok{city\_and\_stops }\OtherTok{\textless{}{-}}\NormalTok{ police[,}\FunctionTok{c}\NormalTok{(}\StringTok{"city"}\NormalTok{, }\StringTok{"traffic\_stops"}\NormalTok{)]}
\NormalTok{city\_and\_stops}
\end{Highlighting}
\end{Shaded}

\begin{verbatim}
##   city traffic_stops
## 1   SF          8814
## 2   DC          9915
## 3   DC          9829
## 4   DC         10161
## 5   SF          6810
## 6   SF          8991
\end{verbatim}

\begin{quote}
Keep in mind that redefining a variable will overwrite it each time, as
we are doing here.
\end{quote}

We can do the same thing for rows by adding a vector of the row indices
to include. For example, to keep just rows 1, 2, and 3 along with
columns ``city'' and ``traffic\_stops'' we could type:

\begin{Shaded}
\begin{Highlighting}[]
\NormalTok{subset1 }\OtherTok{\textless{}{-}}\NormalTok{ police[}\DecValTok{1}\SpecialCharTok{:}\DecValTok{3}\NormalTok{, }\FunctionTok{c}\NormalTok{(}\StringTok{"city"}\NormalTok{, }\StringTok{"traffic\_stops"}\NormalTok{)]}
\NormalTok{subset1}
\end{Highlighting}
\end{Shaded}

\begin{verbatim}
##   city traffic_stops
## 1   SF          8814
## 2   DC          9915
## 3   DC          9829
\end{verbatim}

Or, to keep rows 1, 2, and 4 along with ``city'' and ``traffic\_stops''
columns:

\begin{Shaded}
\begin{Highlighting}[]
\NormalTok{subset2 }\OtherTok{\textless{}{-}}\NormalTok{ police[}\FunctionTok{c}\NormalTok{(}\DecValTok{1}\NormalTok{,}\DecValTok{2}\NormalTok{,}\DecValTok{4}\NormalTok{), }\FunctionTok{c}\NormalTok{(}\StringTok{"city"}\NormalTok{, }\StringTok{"traffic\_stops"}\NormalTok{)]}
\NormalTok{subset2}
\end{Highlighting}
\end{Shaded}

\begin{verbatim}
##   city traffic_stops
## 1   SF          8814
## 2   DC          9915
## 4   DC         10161
\end{verbatim}

Subset by logical condition by using the logical operators discussed
above: \texttt{==}, \texttt{\textgreater{}}, \texttt{\textless{}=}, etc.

For example, if you want to subset only rows with stops less than 9000
you would combine the dollar sign operator along with bracket notation.

This performs a row subsetting operation based on the condition of a
column. Note that the column position is left blank after the comma to
indicate all columns should be retured.

\begin{Shaded}
\begin{Highlighting}[]
\NormalTok{low\_stops }\OtherTok{\textless{}{-}}\NormalTok{ police[police}\SpecialCharTok{$}\NormalTok{traffic\_stops }\SpecialCharTok{\textless{}} \DecValTok{9000}\NormalTok{, ]}
\NormalTok{low\_stops}
\end{Highlighting}
\end{Shaded}

\begin{verbatim}
##   city traffic_stops year
## 1   SF          8814 2000
## 5   SF          6810 2001
## 6   SF          8991 2002
\end{verbatim}

Or, to include multiple conditions use logical and \texttt{\&} (all
conditions must be satisfied) and logical or \texttt{\textbar{}} (just
one condition must be satisfied).

To subset just rows that contain SF as the city and stops less than
7000, type

\begin{Shaded}
\begin{Highlighting}[]
\NormalTok{sf\_low\_stops }\OtherTok{\textless{}{-}}\NormalTok{ police[police}\SpecialCharTok{$}\NormalTok{city }\SpecialCharTok{==} \StringTok{"SF"} \SpecialCharTok{\&}\NormalTok{ police}\SpecialCharTok{$}\NormalTok{traffic\_stops }\SpecialCharTok{\textless{}} \DecValTok{7000}\NormalTok{, ]}
\NormalTok{sf\_low\_stops}
\end{Highlighting}
\end{Shaded}

\begin{verbatim}
##   city traffic_stops year
## 5   SF          6810 2001
\end{verbatim}

\hypertarget{challenge---subsetting}{%
\section{Challenge - subsetting}\label{challenge---subsetting}}

\begin{enumerate}
\def\labelenumi{\arabic{enumi}.}
\item
  Create a subset that contains data from DC or stops less than or equal
  to 7000 and just columns ``city'' and ``traffic\_stops''
\item
  Advanced: use the \texttt{filter()} and \texttt{select()} functions
  from the \texttt{dplyr} R package to do the same thing.
\end{enumerate}

\begin{Shaded}
\begin{Highlighting}[]
\DocumentationTok{\#\# YOUR CODE HERE}
\end{Highlighting}
\end{Shaded}

\hypertarget{getting-help}{%
\subsection{Getting help}\label{getting-help}}

R provides excellent help documentation that often tells you everything
to know, but you might not know it yet!

Type a question mark before a function name to view the help pages.

Read the \textbf{Description} section to understand what the function
does.

The \textbf{Usage} section describes the parameters that you can pass
arguments to.

The \textbf{Arguments} list defines how the arguments work.

Often included (but not always) below these sections are
\textbf{Detiails} that offer more information, \textbf{Value} that
describes the returned objects, \textbf{Notes}, \textbf{Authors},
\textbf{References} and copy/paste \textbf{Examples} to experiment with.

\begin{Shaded}
\begin{Highlighting}[]
\NormalTok{?c}
\NormalTok{?data.frame}

\CommentTok{\# look at help pages for linear regression}
\NormalTok{?lm}

\CommentTok{\# generalized linear models}
\NormalTok{?glm}

\CommentTok{\# arithmetic mean}
\NormalTok{?mean}

\CommentTok{\# histogram}
\NormalTok{?hist}

\CommentTok{\# Wrap symbols in quotations to view their help files}
\NormalTok{?}\StringTok{"\textgreater{}"}
\NormalTok{?}\StringTok{"\&"}
\end{Highlighting}
\end{Shaded}

\hypertarget{putting-it-all-together-css-research-workflow-example}{%
\section{Putting it all together: CSS research workflow
example}\label{putting-it-all-together-css-research-workflow-example}}

Now that you have at least some basic familiarity with basic R syntax
and functionality, let's work through a computational workflow example.

\hypertarget{package-installations}{%
\subsection{Package installations}\label{package-installations}}

Because R has evolved for statistical functionality, it has considerable
variety in its built-in functionality. This means that you can
accomplish many tasks without having to install additional user-defined
software packages that contain shortcuts for virtually every discipline
and that make R work smarter for you.

However, you will likely find yourself needing functionality beyond what
is built into R's base installation.

\begin{Shaded}
\begin{Highlighting}[]
\CommentTok{\# Step 1. Physically download the files to your computer (Internet connection required)}
\FunctionTok{install.packages}\NormalTok{(}\StringTok{"psych"}\NormalTok{)}
\end{Highlighting}
\end{Shaded}

\begin{Shaded}
\begin{Highlighting}[]
\CommentTok{\# Step 2. Import the package\textquotesingle{}s functionality with the library() function and link it to your current RStudio session (do this each time you close and reopen RStudio)}
\FunctionTok{library}\NormalTok{(psych)}
\end{Highlighting}
\end{Shaded}

\begin{Shaded}
\begin{Highlighting}[]
\CommentTok{\# you can also call help on some packages to view vignette\textquotesingle{}s, or purposeful coding walkthroughs with double question marks}
\CommentTok{\# Click "Introduction to the psych package {-} PDF" to learn more}
\NormalTok{??psych}

\NormalTok{?describe}
\NormalTok{?describeBy}
\end{Highlighting}
\end{Shaded}

\begin{Shaded}
\begin{Highlighting}[]
\CommentTok{\# summary statistics for pooled police}
\FunctionTok{describe}\NormalTok{(police)}
\end{Highlighting}
\end{Shaded}

\begin{verbatim}
##               vars n    mean      sd median trimmed    mad  min   max range
## city*            1 6    1.50    0.55    1.5    1.50   0.74    1     2     1
## traffic_stops    2 6 9086.67 1237.59 9410.0 9086.67 816.17 6810 10161  3351
## year             3 6 2001.00    0.89 2001.0 2001.00   1.48 2000  2002     2
##                skew kurtosis     se
## city*          0.00    -2.31   0.22
## traffic_stops -0.84    -0.94 505.24
## year           0.00    -1.96   0.37
\end{verbatim}

\begin{Shaded}
\begin{Highlighting}[]
\CommentTok{\# summary statistics for police, by group }
\FunctionTok{describeBy}\NormalTok{(police, }\AttributeTok{group =}\NormalTok{ police}\SpecialCharTok{$}\NormalTok{city)}
\end{Highlighting}
\end{Shaded}

\begin{verbatim}
## 
##  Descriptive statistics by group 
## group: DC
##               vars n    mean     sd median trimmed    mad  min   max range skew
## city*            1 3    1.00   0.00      1    1.00   0.00    1     1     0  NaN
## traffic_stops    2 3 9968.33 172.31   9915 9968.33 127.50 9829 10161   332 0.28
## year             3 3 2001.00   1.00   2001 2001.00   1.48 2000  2002     2 0.00
##               kurtosis    se
## city*              NaN  0.00
## traffic_stops    -2.33 99.48
## year             -2.33  0.58
## ------------------------------------------------------------ 
## group: SF
##               vars n mean      sd median trimmed    mad  min  max range  skew
## city*            1 3    2    0.00      2       2   0.00    2    2     0   NaN
## traffic_stops    2 3 8205 1211.34   8814    8205 262.42 6810 8991  2181 -0.38
## year             3 3 2001    1.00   2001    2001   1.48 2000 2002     2  0.00
##               kurtosis     se
## city*              NaN   0.00
## traffic_stops    -2.33 699.37
## year             -2.33   0.58
\end{verbatim}

If you have multiple packages to install, consider vectorizing them.

Alternatively, package managers such as
\href{http://trinker.github.io/pacman/vignettes/Introduction_to_pacman.html}{pacman}
provide increased and friendly usability.

\begin{quote}
NOTE: type no in your console and press the Return key if prompted with
``Do you want to install from sources the package which needs
compilation? (Yes/no/cancel)''. If asked to update packages, type 1 in
your console and press the Return key unless you have specific
\end{quote}

\begin{Shaded}
\begin{Highlighting}[]
\CommentTok{\# Step 1. Install the files }
\FunctionTok{install.packages}\NormalTok{(}\FunctionTok{c}\NormalTok{(}\StringTok{"psych"}\NormalTok{, }\StringTok{"tidyr"}\NormalTok{, }\StringTok{"dplyr"}\NormalTok{, }\StringTok{"ggplot2"}\NormalTok{, }\StringTok{"broom"}\NormalTok{, }\StringTok{"tidycensus"}\NormalTok{))}

\CommentTok{\# Step 2. Import them to your current session}
\FunctionTok{library}\NormalTok{(psych)}
\FunctionTok{library}\NormalTok{(tidyr)}
\FunctionTok{library}\NormalTok{(dplyr)}
\FunctionTok{library}\NormalTok{(ggplot2)}
\FunctionTok{library}\NormalTok{(broom)}
\FunctionTok{library}\NormalTok{(tidycensus)}
\end{Highlighting}
\end{Shaded}

\hypertarget{import-data}{%
\subsection{Import data}\label{import-data}}

Import the Mesa, Arizona, USA, dataset from The Stanford Open Policing
Project.

\begin{Shaded}
\begin{Highlighting}[]
\NormalTok{mesa }\OtherTok{\textless{}{-}} \FunctionTok{read.csv}\NormalTok{(}\StringTok{"data/raw/az\_mesa\_2020\_04\_01.csv"}\NormalTok{, }
                 \AttributeTok{stringsAsFactors =} \ConstantTok{TRUE}\NormalTok{)}
\FunctionTok{str}\NormalTok{(mesa)}
\end{Highlighting}
\end{Shaded}

\begin{verbatim}
## 'data.frame':    96621 obs. of  19 variables:
##  $ raw_row_number     : int  1 2 3 4 5 6 7 8 9 10 ...
##  $ date               : Factor w/ 1186 levels "2014-01-01","2014-01-02",..: 1 1 1 1 1 1 1 1 1 1 ...
##  $ time               : Factor w/ 1440 levels "00:00:00","00:01:00",..: 305 754 748 833 183 93 968 1078 90 474 ...
##  $ location           : Factor w/ 10830 levels "#VALUE!, MESA",..: 7553 5263 5072 5372 3504 2212 2184 2184 3149 4202 ...
##  $ lat                : num  33.4 33.4 33.4 33.4 33.4 ...
##  $ lng                : num  -112 -112 -112 -112 -112 ...
##  $ subject_age        : int  36 58 19 46 27 58 24 27 27 29 ...
##  $ subject_race       : Factor w/ 6 levels "asian/pacific islander",..: 3 6 6 2 6 6 6 6 6 3 ...
##  $ subject_sex        : Factor w/ 2 levels "female","male": 1 1 2 2 2 2 2 2 2 1 ...
##  $ officer_id_hash    : Factor w/ 770 levels "0027657d5f","003bde5bfa",..: 637 677 226 529 272 272 754 754 128 146 ...
##  $ type               : Factor w/ 2 levels "pedestrian","vehicular": NA 2 2 2 2 2 NA NA 2 2 ...
##  $ violation          : Factor w/ 716 levels "2 OCCUPANTS UNDER 18 CLASS G LICENSE",..: 405 391 615 640 125 157 225 595 316 65 ...
##  $ arrest_made        : logi  FALSE FALSE FALSE FALSE FALSE FALSE ...
##  $ citation_issued    : logi  TRUE TRUE TRUE TRUE TRUE TRUE ...
##  $ warning_issued     : logi  FALSE FALSE FALSE FALSE FALSE FALSE ...
##  $ outcome            : Factor w/ 2 levels "arrest","citation": 2 2 2 2 2 2 1 1 2 2 ...
##  $ raw_race_fixed     : Factor w/ 5 levels "A","B","I","U",..: 5 5 5 2 5 5 5 5 5 5 ...
##  $ raw_ethnicity_fixed: Factor w/ 3 levels "H","N","U": 1 2 2 2 2 2 2 2 2 1 ...
##  $ raw_charge         : Factor w/ 66 levels "-----","------",..: 33 9 32 7 5 5 4 6 44 32 ...
\end{verbatim}

\begin{Shaded}
\begin{Highlighting}[]
\CommentTok{\# view the first six rows by default}
\FunctionTok{head}\NormalTok{(mesa)}
\end{Highlighting}
\end{Shaded}

\begin{verbatim}
##   raw_row_number       date     time                    location      lat
## 1              1 2014-01-01 05:04:00    700 E FRANKLIN AVE, MESA 33.40238
## 2              2 2014-01-01 12:33:00        400 E BROWN RD, MESA 33.43558
## 3              3 2014-01-01 12:27:00   3600 E SOUTHERN AVE, MESA 33.39370
## 4              4 2014-01-01 13:52:00       400 N CENTER ST, MESA 33.42234
## 5              5 2014-01-01 03:02:00 2100 W UNIVERSITY0 DR, MESA 33.42212
## 6              6 2014-01-01 01:32:00      1600 S VAL VISTA, MESA 33.38605
##         lng subject_age subject_race subject_sex officer_id_hash      type
## 1 -111.8161          36     hispanic      female      d4eed1c4dc      <NA>
## 2 -111.8227          58        white      female      e24408e0fd vehicular
## 3 -111.7538          19        white        male      4ab12ca221 vehicular
## 4 -111.8314          46        black        male      b1c5205466 vehicular
## 5 -111.8764          27        white        male      5cbad8bdf3 vehicular
## 6 -111.7540          58        white        male      5cbad8bdf3 vehicular
##                                      violation arrest_made citation_issued
## 1                      NOISE AFTER 10PM TO 6AM       FALSE            TRUE
## 2                        NO PROOF OF INSURANCE       FALSE            TRUE
## 3 SPEED NOT R&P/FTC SPEED TO AVOID A COLLISION       FALSE            TRUE
## 4                             SUSP/REVOKED LIC       FALSE            TRUE
## 5        DRIVE ON SUSPENDED OR REVOKED LICENSE       FALSE            TRUE
## 6                         EXPIRED REGISTRATION       FALSE            TRUE
##   warning_issued  outcome raw_race_fixed raw_ethnicity_fixed raw_charge
## 1          FALSE citation              W                   H   ------A-
## 2          FALSE citation              W                   N   -------C
## 3          FALSE citation              W                   N    ------A
## 4          FALSE citation              B                   N   -------B
## 5          FALSE citation              W                   N   -------A
## 6          FALSE citation              W                   N   -------A
\end{verbatim}

\begin{Shaded}
\begin{Highlighting}[]
\CommentTok{\# count number of missing values}
\FunctionTok{sum}\NormalTok{(}\FunctionTok{is.na}\NormalTok{(mesa))}
\end{Highlighting}
\end{Shaded}

\begin{verbatim}
## [1] 13324
\end{verbatim}

\begin{Shaded}
\begin{Highlighting}[]
\CommentTok{\# calculate proportion of missing data}
\NormalTok{(}\FunctionTok{sum}\NormalTok{(}\FunctionTok{is.na}\NormalTok{(mesa))) }\SpecialCharTok{/}\NormalTok{ (}\FunctionTok{nrow}\NormalTok{(mesa) }\SpecialCharTok{*} \FunctionTok{ncol}\NormalTok{(mesa))}
\end{Highlighting}
\end{Shaded}

\begin{verbatim}
## [1] 0.007257875
\end{verbatim}

\hypertarget{wrangle}{%
\subsection{Wrangle}\label{wrangle}}

Subset only columns: date, subject\_race, subject\_sex, subject\_age,
officer\_id\_hash, violation, arrest\_made

\begin{Shaded}
\begin{Highlighting}[]
\NormalTok{clean }\OtherTok{\textless{}{-}}\NormalTok{ mesa[, }\FunctionTok{c}\NormalTok{(}\StringTok{"date"}\NormalTok{, }\StringTok{"subject\_race"}\NormalTok{, }\StringTok{"subject\_sex"}\NormalTok{, }\StringTok{"subject\_age"}\NormalTok{, }\StringTok{"officer\_id\_hash"}\NormalTok{, }\StringTok{"violation"}\NormalTok{, }\StringTok{"arrest\_made"}\NormalTok{)]}
\FunctionTok{str}\NormalTok{(clean)}
\end{Highlighting}
\end{Shaded}

\begin{verbatim}
## 'data.frame':    96621 obs. of  7 variables:
##  $ date           : Factor w/ 1186 levels "2014-01-01","2014-01-02",..: 1 1 1 1 1 1 1 1 1 1 ...
##  $ subject_race   : Factor w/ 6 levels "asian/pacific islander",..: 3 6 6 2 6 6 6 6 6 3 ...
##  $ subject_sex    : Factor w/ 2 levels "female","male": 1 1 2 2 2 2 2 2 2 1 ...
##  $ subject_age    : int  36 58 19 46 27 58 24 27 27 29 ...
##  $ officer_id_hash: Factor w/ 770 levels "0027657d5f","003bde5bfa",..: 637 677 226 529 272 272 754 754 128 146 ...
##  $ violation      : Factor w/ 716 levels "2 OCCUPANTS UNDER 18 CLASS G LICENSE",..: 405 391 615 640 125 157 225 595 316 65 ...
##  $ arrest_made    : logi  FALSE FALSE FALSE FALSE FALSE FALSE ...
\end{verbatim}

\begin{Shaded}
\begin{Highlighting}[]
\CommentTok{\# or}

\NormalTok{clean }\OtherTok{\textless{}{-}}\NormalTok{ mesa[, }\FunctionTok{c}\NormalTok{(}\DecValTok{2}\NormalTok{, }\DecValTok{8}\NormalTok{, }\DecValTok{9}\NormalTok{, }\DecValTok{7}\NormalTok{, }\DecValTok{10}\NormalTok{, }\DecValTok{12}\NormalTok{, }\DecValTok{13}\NormalTok{)]}
\FunctionTok{str}\NormalTok{(clean)}
\end{Highlighting}
\end{Shaded}

\begin{verbatim}
## 'data.frame':    96621 obs. of  7 variables:
##  $ date           : Factor w/ 1186 levels "2014-01-01","2014-01-02",..: 1 1 1 1 1 1 1 1 1 1 ...
##  $ subject_race   : Factor w/ 6 levels "asian/pacific islander",..: 3 6 6 2 6 6 6 6 6 3 ...
##  $ subject_sex    : Factor w/ 2 levels "female","male": 1 1 2 2 2 2 2 2 2 1 ...
##  $ subject_age    : int  36 58 19 46 27 58 24 27 27 29 ...
##  $ officer_id_hash: Factor w/ 770 levels "0027657d5f","003bde5bfa",..: 637 677 226 529 272 272 754 754 128 146 ...
##  $ violation      : Factor w/ 716 levels "2 OCCUPANTS UNDER 18 CLASS G LICENSE",..: 405 391 615 640 125 157 225 595 316 65 ...
##  $ arrest_made    : logi  FALSE FALSE FALSE FALSE FALSE FALSE ...
\end{verbatim}

\hypertarget{summarize}{%
\subsection{Summarize}\label{summarize}}

Get some quick facts about your data with the below functions

\begin{Shaded}
\begin{Highlighting}[]
\CommentTok{\# Tabulate frequencies of a single variable }
\FunctionTok{table}\NormalTok{(clean}\SpecialCharTok{$}\NormalTok{subject\_race)}
\end{Highlighting}
\end{Shaded}

\begin{verbatim}
## 
## asian/pacific islander                  black               hispanic 
##                   1278                   6459                  16626 
##                  other                unknown                  white 
##                   2305                   7092                  62861
\end{verbatim}

\begin{Shaded}
\begin{Highlighting}[]
\FunctionTok{table}\NormalTok{(clean}\SpecialCharTok{$}\NormalTok{arrest\_made)}
\end{Highlighting}
\end{Shaded}

\begin{verbatim}
## 
## FALSE  TRUE 
## 95113  1508
\end{verbatim}

\begin{Shaded}
\begin{Highlighting}[]
\CommentTok{\# Compute crosstabs}
\FunctionTok{table}\NormalTok{(clean}\SpecialCharTok{$}\NormalTok{subject\_race, clean}\SpecialCharTok{$}\NormalTok{arrest\_made)}
\end{Highlighting}
\end{Shaded}

\begin{verbatim}
##                         
##                          FALSE  TRUE
##   asian/pacific islander  1268    10
##   black                   6255   204
##   hispanic               16181   445
##   other                   2183   122
##   unknown                 7088     4
##   white                  62138   723
\end{verbatim}

\begin{Shaded}
\begin{Highlighting}[]
\CommentTok{\# Produce summary statistics}
\CommentTok{\# Six number summaries (minimum and maximum values, 1st and 3rd quartile boundaries, median, and mean) are returned for quantitative variables (i.e., subject\_age)}
\CommentTok{\# Frequencies are returned for non quantitative variables (all other columns)}
\FunctionTok{summary}\NormalTok{(clean)}
\end{Highlighting}
\end{Shaded}

\begin{verbatim}
##          date                       subject_race   subject_sex   
##  2015-05-28:  224   asian/pacific islander: 1278   female:39083  
##  2017-03-30:  212   black                 : 6459   male  :56248  
##  2015-08-20:  202   hispanic              :16626   NA's  : 1290  
##  2015-05-19:  197   other                 : 2305                 
##  2015-03-02:  188   unknown               : 7092                 
##  2015-04-16:  186   white                 :62861                 
##  (Other)   :95412                                                
##   subject_age      officer_id_hash 
##  Min.   :10.00   cb50401ff4: 3563  
##  1st Qu.:25.00   2ea1bea91b: 2599  
##  Median :34.00   aea894f584: 2594  
##  Mean   :37.41   cf084c69a3: 2571  
##  3rd Qu.:47.00   29dac4e72e: 2550  
##  Max.   :99.00   96fcddf228: 2467  
##  NA's   :1448    (Other)   :80277  
##                                         violation     arrest_made    
##  SPEED NOT R&P/FTC SPEED TO AVOID A COLLISION:30021   Mode :logical  
##  EXPIRED REGISTRATION                        : 6622   FALSE:95113    
##  NO PROOF OF INSURANCE                       : 4547   TRUE :1508     
##  M / I SUSPENSION REGISTRATION               : 3624                  
##  NO VALID DRIVERS LICENSE - IN/OUT OF STATE  : 3548                  
##  (Other)                                     :48211                  
##  NA's                                        :   48
\end{verbatim}

\begin{Shaded}
\begin{Highlighting}[]
\CommentTok{\# use the describeBy() function from the psych package}
\CommentTok{\# by sex}
\NormalTok{summary\_sex }\OtherTok{\textless{}{-}} \FunctionTok{describeBy}\NormalTok{(clean, }\AttributeTok{group =}\NormalTok{ clean}\SpecialCharTok{$}\NormalTok{subject\_sex)}
\end{Highlighting}
\end{Shaded}

\begin{verbatim}
## Warning in FUN(newX[, i], ...): no non-missing arguments to min; returning Inf
\end{verbatim}

\begin{verbatim}
## Warning in FUN(newX[, i], ...): no non-missing arguments to max; returning -Inf
\end{verbatim}

\begin{verbatim}
## Warning in FUN(newX[, i], ...): no non-missing arguments to min; returning Inf
\end{verbatim}

\begin{verbatim}
## Warning in FUN(newX[, i], ...): no non-missing arguments to max; returning -Inf
\end{verbatim}

\begin{Shaded}
\begin{Highlighting}[]
\NormalTok{summary\_sex}
\end{Highlighting}
\end{Shaded}

\begin{verbatim}
## 
##  Descriptive statistics by group 
## group: female
##                  vars     n   mean     sd median trimmed    mad min  max range
## date*               1 39083 565.99 331.20    558  560.09 389.92   1 1186  1185
## subject_race*       2 39083   5.10   1.45      6    5.36   0.00   1    6     5
## subject_sex*        3 39083   1.00   0.00      1    1.00   0.00   1    1     0
## subject_age         4 38960  37.69  15.45     34   36.10  14.83  14   99    85
## officer_id_hash*    5 39083 405.43 227.91    434  410.13 289.11   1  770   769
## violation*          6 39063 432.55 196.65    451  446.29 243.15   1  716   715
## arrest_made         7 39083    NaN     NA     NA     NaN     NA Inf -Inf  -Inf
##                   skew kurtosis   se
## date*             0.12    -1.01 1.68
## subject_race*    -1.24    -0.01 0.01
## subject_sex*       NaN      NaN 0.00
## subject_age       0.85     0.18 0.08
## officer_id_hash* -0.14    -1.32 1.15
## violation*       -0.39    -1.32 0.99
## arrest_made         NA       NA   NA
## ------------------------------------------------------------ 
## group: male
##                  vars     n   mean     sd median trimmed    mad min  max range
## date*               1 56248 557.91 331.74    548  550.68 392.89   1 1186  1185
## subject_race*       2 56248   4.98   1.51      6    5.22   0.00   1    6     5
## subject_sex*        3 56248   2.00   0.00      2    2.00   0.00   2    2     0
## subject_age         4 56101  37.20  15.47     34   35.58  16.31  10   99    89
## officer_id_hash*    5 56248 400.74 226.67    432  404.40 286.14   1  770   769
## violation*          6 56220 408.29 203.45    392  419.45 330.62   1  711   710
## arrest_made         7 56248    NaN     NA     NA     NaN     NA Inf -Inf  -Inf
##                   skew kurtosis   se
## date*             0.14    -1.02 1.40
## subject_race*    -1.03    -0.57 0.01
## subject_sex*       NaN      NaN 0.00
## subject_age       0.85     0.10 0.07
## officer_id_hash* -0.10    -1.31 0.96
## violation*       -0.26    -1.37 0.86
## arrest_made         NA       NA   NA
\end{verbatim}

\begin{Shaded}
\begin{Highlighting}[]
\CommentTok{\# Or, just access the female data, and subset just subject\_age and columns 3 and 4 (mean and sd)}
\NormalTok{mesa\_female }\OtherTok{\textless{}{-}}\NormalTok{ summary\_sex}\SpecialCharTok{$}\NormalTok{female[}\StringTok{"subject\_age"}\NormalTok{ ,}\DecValTok{3}\SpecialCharTok{:}\DecValTok{4}\NormalTok{]}
\NormalTok{mesa\_female}
\end{Highlighting}
\end{Shaded}

\begin{verbatim}
##              mean    sd
## subject_age 37.69 15.45
\end{verbatim}

\begin{Shaded}
\begin{Highlighting}[]
\CommentTok{\# or}
\NormalTok{mesa\_female }\OtherTok{\textless{}{-}}\NormalTok{ summary\_sex}\SpecialCharTok{$}\NormalTok{female[}\StringTok{"subject\_age"}\NormalTok{, }\FunctionTok{c}\NormalTok{(}\StringTok{"mean"}\NormalTok{, }\StringTok{"sd"}\NormalTok{)]}
\NormalTok{mesa\_female}
\end{Highlighting}
\end{Shaded}

\begin{verbatim}
##              mean    sd
## subject_age 37.69 15.45
\end{verbatim}

\hypertarget{visualize}{%
\subsection{Visualize}\label{visualize}}

Perhaps there are differences in sex, race, and age within our data.
Let's eyeball it to find out!

A histogram can be used to view the distribution of one variable.

\begin{Shaded}
\begin{Highlighting}[]
\FunctionTok{hist}\NormalTok{(clean}\SpecialCharTok{$}\NormalTok{subject\_age)}
\end{Highlighting}
\end{Shaded}

\includegraphics{Yen_Part1_files/figure-latex/unnamed-chunk-32-1.pdf}

\begin{Shaded}
\begin{Highlighting}[]
\FunctionTok{hist}\NormalTok{(clean}\SpecialCharTok{$}\NormalTok{subject\_age, }
     \AttributeTok{breaks =} \DecValTok{5}\NormalTok{, }
     \AttributeTok{main =} \StringTok{"Include a title"}\NormalTok{)}
\end{Highlighting}
\end{Shaded}

\includegraphics{Yen_Part1_files/figure-latex/unnamed-chunk-32-2.pdf}

\begin{Shaded}
\begin{Highlighting}[]
\FunctionTok{hist}\NormalTok{(clean}\SpecialCharTok{$}\NormalTok{subject\_age, }
     \AttributeTok{breaks =} \DecValTok{50}\NormalTok{, }
     \AttributeTok{col =} \StringTok{"gray80"}\NormalTok{,}
     \AttributeTok{main =} \StringTok{"Age distribution of Mesa, AZ USA traffic stops}\SpecialCharTok{\textbackslash{}n}\StringTok{ December 2{-}13 {-} March 2017"}\NormalTok{,}
     \AttributeTok{xlab =} \StringTok{"Driver Age"}\NormalTok{,}
     \AttributeTok{ylab =} \StringTok{"Frequency"}\NormalTok{, }
     \AttributeTok{las =} \DecValTok{1}\NormalTok{)}
\end{Highlighting}
\end{Shaded}

\includegraphics{Yen_Part1_files/figure-latex/unnamed-chunk-32-3.pdf}

\begin{Shaded}
\begin{Highlighting}[]
\CommentTok{\# Type colors() to see the 657 stock colors available }
\CommentTok{\# colors()}
\end{Highlighting}
\end{Shaded}

The data are positively skewed and we know from our summary statistics
that the mean is \textasciitilde37 years old and is widely dispersed.

Boxplots can be used similarly to histograms to view the distribution of
one variable, but we can conveniently plot categorical groupings side by
side for interpretation.

\begin{Shaded}
\begin{Highlighting}[]
\CommentTok{\# subject\_sex}
\FunctionTok{boxplot}\NormalTok{(clean}\SpecialCharTok{$}\NormalTok{subject\_age }\SpecialCharTok{\textasciitilde{}}\NormalTok{ clean}\SpecialCharTok{$}\NormalTok{subject\_sex)}
\end{Highlighting}
\end{Shaded}

\includegraphics{Yen_Part1_files/figure-latex/unnamed-chunk-33-1.pdf}

\begin{Shaded}
\begin{Highlighting}[]
\CommentTok{\# subject\_race}
\FunctionTok{boxplot}\NormalTok{(clean}\SpecialCharTok{$}\NormalTok{subject\_age }\SpecialCharTok{\textasciitilde{}}\NormalTok{ clean}\SpecialCharTok{$}\NormalTok{subject\_race)}
\end{Highlighting}
\end{Shaded}

\includegraphics{Yen_Part1_files/figure-latex/unnamed-chunk-33-2.pdf}

\hypertarget{analyze}{%
\subsection{Analyze}\label{analyze}}

We can now ask a question such as: are there statistically significant
differences in arrests made by race? Something like a one-way analysis
of variance (ANOVA) model can help us understand.

\begin{Shaded}
\begin{Highlighting}[]
\NormalTok{aov\_race }\OtherTok{\textless{}{-}} \FunctionTok{aov}\NormalTok{(clean}\SpecialCharTok{$}\NormalTok{arrest\_made }\SpecialCharTok{\textasciitilde{}}\NormalTok{ clean}\SpecialCharTok{$}\NormalTok{subject\_race)}
\FunctionTok{summary}\NormalTok{(aov\_race)}
\end{Highlighting}
\end{Shaded}

\begin{verbatim}
##                       Df Sum Sq Mean Sq F value Pr(>F)    
## clean$subject_race     5    9.7  1.9342   126.7 <2e-16 ***
## Residuals          96615 1474.8  0.0153                   
## ---
## Signif. codes:  0 '***' 0.001 '**' 0.01 '*' 0.05 '.' 0.1 ' ' 1
\end{verbatim}

This p-value indicates that there are statistically significant
differences by race. But, how do we know between which groups? We can
use a post-hoc test to find out the pairwise differences, along with
adjusted p-values.

\begin{Shaded}
\begin{Highlighting}[]
\FunctionTok{TukeyHSD}\NormalTok{(aov\_race)}
\end{Highlighting}
\end{Shaded}

\begin{verbatim}
##   Tukey multiple comparisons of means
##     95% family-wise confidence level
## 
## Fit: aov(formula = clean$arrest_made ~ clean$subject_race)
## 
## $`clean$subject_race`
##                                         diff          lwr           upr
## black-asian/pacific islander     0.023759110  0.012980037  0.0345381842
## hispanic-asian/pacific islander  0.018940581  0.008720392  0.0291607704
## other-asian/pacific islander     0.045103690  0.032824613  0.0573827675
## unknown-asian/pacific islander  -0.007260710 -0.017960030  0.0034386091
## white-asian/pacific islander     0.003676841 -0.006271445  0.0136251263
## hispanic-black                  -0.004818529 -0.009980691  0.0003436328
## other-black                      0.021344580  0.012802251  0.0298869090
## unknown-black                   -0.031019821 -0.037075486 -0.0249641551
## white-black                     -0.020082270 -0.024682708 -0.0154818313
## other-hispanic                   0.026163109  0.018337816  0.0339884027
## unknown-hispanic                -0.026201292 -0.031194779 -0.0212078037
## white-hispanic                  -0.015263740 -0.018334224 -0.0121932571
## unknown-other                   -0.052364401 -0.060805869 -0.0439229322
## white-other                     -0.041426850 -0.048893531 -0.0339601678
## white-unknown                    0.010937551  0.006527218  0.0153478842
##                                     p adj
## black-asian/pacific islander    0.0000000
## hispanic-asian/pacific islander 0.0000019
## other-asian/pacific islander    0.0000000
## unknown-asian/pacific islander  0.3812921
## white-asian/pacific islander    0.8996599
## hispanic-black                  0.0834401
## other-black                     0.0000000
## unknown-black                   0.0000000
## white-black                     0.0000000
## other-hispanic                  0.0000000
## unknown-hispanic                0.0000000
## white-hispanic                  0.0000000
## unknown-other                   0.0000000
## white-other                     0.0000000
## white-unknown                   0.0000000
\end{verbatim}

Read the du Prel et al.~article ``Confidence Interval or P-Value''
linked in the README file to learn more about hypothesis testing, point
estimations, and confidence in analysis.

\hypertarget{interpret}{%
\subsection{Interpret}\label{interpret}}

Results of a one-way ANOVA could be interpreted to suggest that white
drivers are stopped less than all other groups. Why might this be true?
Is there bias in our data? Do sample size differences influence the
results? Did we violate any
\href{http://www.sthda.com/english/wiki/one-way-anova-test-in-r}{assumptions}
of the test? Would a non-parametric alternative such as
\texttt{kruskal.test()} be more appropriate? These are the tough
questions that are up to the researcher to investigate.

\hypertarget{save-your-work}{%
\section{Save your work}\label{save-your-work}}

\begin{quote}
BINDER USERS! CLick the ``More'' button on the ``Files'' tab in Binder,
and then select ``Export'' to download files or folders to your
computer. Screenshot below:
\end{quote}

\begin{figure}
\centering
\includegraphics{img/binder_export.png}
\caption{binder\_export}
\end{figure}

There are many ways to import and export data in R.

We can save datasets and tables we create with the \texttt{write.csv()}
function. The first argument is the variable name, the second argument
is the file path/name.

\begin{quote}
Note that we save this in the ``preprocessed'' data subfolder, and could
even save it in a separate directory named ``Tables'' or something like
that.
\end{quote}

\begin{Shaded}
\begin{Highlighting}[]
\FunctionTok{write.csv}\NormalTok{(police, }\AttributeTok{file =} \StringTok{"data/preprocessed/police.csv"}\NormalTok{)}
\end{Highlighting}
\end{Shaded}

Where does it save? Use \texttt{getwd()} to find out. Also check out the
\href{https://cran.r-project.org/web/packages/here/vignettes/here.html}{here
R package vignette} to learn more efficient ways of structuring and
understanding

We specified the ``data'' --\textgreater{} ``preprocessed'' folder of
the parent directory ``sicss-howard-mathematica-R-2022'' folder.

\begin{Shaded}
\begin{Highlighting}[]
\FunctionTok{getwd}\NormalTok{()}
\end{Highlighting}
\end{Shaded}

\begin{verbatim}
## [1] "C:/Users/BJY/OneDrive - Indiana University/@ IU 2022 Summer/Mathematica_Howard/sicss-howard-mathematica-R-2022"
\end{verbatim}

We can also use the \texttt{save()} function to save variables in an
.RData file for easy loadindg in the next lesson. Save only the
\texttt{clean} dataset like this:

\begin{Shaded}
\begin{Highlighting}[]
\FunctionTok{save}\NormalTok{(clean, }\AttributeTok{file =} \StringTok{"data/preprocessed/clean.RData"}\NormalTok{)}
\end{Highlighting}
\end{Shaded}

\hypertarget{a-note-on-computational-reproducibility}{%
\section{A note on computational
reproducibility}\label{a-note-on-computational-reproducibility}}

``Reproducible'' research is a term often discussed but rarely
understood. It is something you want to build into your research design
very early on! As a project nears completion, it is very difficult to
make it reproducible in retrospect. Ask me if you want to learn more!

\begin{itemize}
\item
  Replication = code + data
\item
  Computational reproduciblity = code + data + environment +
  distribution
\item
  Reproducibility checklist by
  \href{http://www.biostat.jhsph.edu/~rpeng/}{Roger Peng}

  \begin{enumerate}
  \def\labelenumi{\arabic{enumi}.}
  \item
    Start with science (avoid vague questions and concepts)
  \item
    Don't do things by hand (not only about automation but also
    documentation)
  \item
    Don't point and click (same problem)
  \item
    Teach a computer (automation also solves documentation to some
    extent)
  \item
    Use some version control
  \item
    Don't save output (instead, keep the input and code)
  \item
    Set your seed
  \item
    Think about the entire pipeline
  \end{enumerate}
\end{itemize}

\hypertarget{challenge---workflows}{%
\section{Challenge - workflows}\label{challenge---workflows}}

\begin{enumerate}
\def\labelenumi{\arabic{enumi}.}
\item
  Reproduce the above workflow using the dataset
  ``ca\_oakland\_2020\_04\_01 2.csv''.
\item
  Ask the question: how does the distribution of arrests vary by race?
  Hint: use a barplot (you might have to generate frequencies first to
  make the barplot!).
\item
  Save the ``ca\_oakland\_2020\_04\_01 2.csv'' dataset in a variable
  named oak. Use the \texttt{save()} function to save it in a file name
  ``oak.csv'' in the ``data/preprocessed'' folder.
\item
  Advanced: Use ggplot2 to make the barplot.
\end{enumerate}

What other statistical applications might be useful?

\end{document}
